%%% Template originaly created by Karol Kozioł (mail@karol-koziol.net) and modified for ShareLaTeX use

\documentclass[a4paper,11pt]{article}

\usepackage[T1]{fontenc}
\usepackage[utf8]{inputenc}
\usepackage[francais]{babel}
\usepackage{verbatim}
\usepackage{graphicx}
\usepackage{xcolor}

\usepackage[defaultmono]{droidmono}

\usepackage{amsmath,amssymb,amsthm,textcomp}
\usepackage{enumerate}
\usepackage{multicol}
\usepackage{tikz}
\usepackage{hyperref}

\usepackage{geometry}
\geometry{total={210mm,297mm},
left=25mm,right=25mm,%
bindingoffset=0mm, top=20mm,bottom=20mm}


%\linespread{1.3}

\newcommand{\linia}{\rule{\linewidth}{0.5pt}}

% my own titles
\makeatletter
\renewcommand{\maketitle}{
\begin{center}
\vspace{2ex}
{\huge \textsc{\@title}}
\vspace{1ex}
\\
\linia\\
\@author \hfill \@date
\vspace{4ex}
\end{center}
}
\makeatother
%%%

% custom footers and headers
\usepackage{fancyhdr}
\pagestyle{fancy}
\lhead{}
\chead{}
\rhead{}
\lfoot{Correction IS langage C}
\cfoot{}
\rfoot{Page \thepage}
\renewcommand{\headrulewidth}{0pt}
\renewcommand{\footrulewidth}{0pt}
%

% code listing settings
\usepackage{listings}
\lstset{
    language=C,
    basicstyle=\ttfamily\footnotesize,
    aboveskip={1.0\baselineskip},
    belowskip={1.0\baselineskip},
    columns=fixed,
    extendedchars=true,
    breaklines=true,
    tabsize=4,
    prebreak=\raisebox{0ex}[0ex][0ex]{\ensuremath{\hookleftarrow}},
    frame=lines,
    showtabs=false,
    showspaces=false,
    showstringspaces=false,
    keywordstyle=\color[rgb]{0.627,0.126,0.941},
    commentstyle=\color[rgb]{0.133,0.545,0.133},
    stringstyle=\color[rgb]{01,0,0},
    numbers=left,
    numberstyle=\small,
    stepnumber=1,
    numbersep=10pt,
    captionpos=t,
    escapeinside={\%*}{*)}
}
\renewcommand\lstlistingname{Code source}
%%%----------%%%----------%%%----------%%%----------%%%

\begin{document}

\title{Correction IS langage C}

\author{Elliot S.}

\date{2013/2014}

\maketitle

\section*{Question 1}

Il était demandé de compléter le fichier "mots\_meles.h".

\subsection*{1) Macros}

\begin{lstlisting}[caption=Les macros TAILLEGRILLE et LGDICO]
/* macro pour la taille de la grille */
#define TAILLEGRILLE 5

/* macro pour le nombre de mots a trouver */
#define LGDICO 5
\end{lstlisting}

\subsection*{2) Type DIRECTION}

\begin{lstlisting}[caption=Le type DIRECTION]
/* type enumere pour choisir la direction : E=0, W=1, N=2, S=3, NE=4, SE=5, NW=6, SW=7 */
typedef enum 
{ E, W, N, S, NE, SE, NW, SW } DIRECTION;
\end{lstlisting}

\subsection*{3) Type typeMot}

\begin{lstlisting}[caption=Le type typeMot]
/* typeMot pour definir les mots a trouver ainsi que leur longueur et s'ils ont ete trouves ou non */
typedef struct{ 
	int tailleMot;
	char mot[TAILLEGRILLE];
	int trouve; 
} typeMot;
\end{lstlisting}

\newpage{}

\section*{Question 2}

Il s'agissait ici d'écrire quelques fonctions dans le fichier "mots\_meles.c".

\subsection*{1) Fonction verifCoord}
\begin{lstlisting}[caption=La fonction verifCoord]
int verifCoord(int x, int y, int tailleGrille){
	return (x >= 0) && (x < tailleGrille) 
	        && (y >= 0) && (y < tailleGrille);
}
\end{lstlisting}

\subsection*{2) Fonction calculeDirection}
\begin{lstlisting}[caption=La fonction calculeDirection]
typeCoord calculeDirection(DIRECTION dir){
	typeCoord res;
	switch(dir){
		case E : res.x = 0; res.y = 1;
				break;
		case W : res.x = 0; res.y = -1;
				break;
		case N : res.x = -1; res.y = 0;
				break;
		case S : res.x = 1; res.y = 0;
				break;
		case NE : res.x = -1; res.y = 1;
				break;
		case SE : res.x = 1; res.y = 1;
				break;
		case NW : res.x = -1; res.y = -1;
				break;
		case SW : res.x = 1; res.y = -1;
				break;
	}
	return res;
}
\end{lstlisting}

\newpage{}
\subsection*{3) Fonction compareMots}
\begin{lstlisting}[caption=La fonction compareMots]
int compareMots(char mot1[TAILLEGRILLE], int lgMot1, char mot2[TAILLEGRILLE], int lgMot2){
	if(lgMot1 != lgMot2)
		return 0;
	else{
		int i = 0;
		while (i < lgMot1 && mot1[i] == mot2[i])
			i++;
		return (i == lgMot1);
	}
}
\end{lstlisting}

\subsection*{4) Fonction rechercheMot}
\begin{lstlisting}[caption=La fonction rechercheMot]
int rechercheMot(typeMot motsATrouver[LGDICO], char mot[TAILLEGRILLE], int lgMot){
	int i;
	for(i = 0; i<LGDICO; i++)
		if(compareMots(motsATrouver[i].mot, motsATrouver[i].tailleMot, mot, lgMot))
			return i;
	return LGDICO;
}
\end{lstlisting}

\subsection*{5) Fonction extraireMot}
\begin{lstlisting}[caption=La fonction extraireMot]
int extraireMot(char grille[TAILLEGRILLE][TAILLEGRILLE], int x, int y , int dx, int dy, char mot[TAILLEGRILLE], int lgMot){
	int i = 0;
	while(i < lgMot && verifCoord(x + i*dx,y + i*dy,TAILLEGRILLE)){
		mot[i] = grille[x + i*dx][y + i*dy];
		i++;
	}	
	return i;
}
\end{lstlisting}

\newpage{}

\subsection*{6) Fonction afficheGrille}
\begin{lstlisting}[caption=La fonction afficheGrille]
void afficheGrille(char grille[TAILLEGRILLE][TAILLEGRILLE], int grilleTrouve[TAILLEGRILLE][TAILLEGRILLE], typeMot motsATrouver[LGDICO]){
	int i,j;
	printf("mot meles :\n");
	printf("   X ");
	for (i = 0; i < TAILLEGRILLE; i++)
		printf("%d  ", i);
	printf("\n");
	for (i = 0; i < TAILLEGRILLE; i++){
		printf("   %d", i);
		for (j = 0; j < TAILLEGRILLE; j++)
			if(grilleTrouve[i][j])
				printf("(%c)", grille[i][j]);
			else
				printf(" %c ", grille[i][j]);
		printf("\t\t");
		if(!(motsATrouver[i].trouve))
			afficheMot(motsATrouver[i].mot, motsATrouver[i].tailleMot);
		printf("\n");
	}
}
\end{lstlisting}

\subsection*{7) Fonction toutTrouve}
\begin{lstlisting}[caption=La fonction toutTrouve]
int toutTrouve(typeMot motsATrouver[LGDICO]){
	int i = 0;
	while(i < LGDICO && motsATrouver[i].trouve)
		i++;
	return (i == LGDICO);
}
\end{lstlisting}

\newpage{}

\section*{Question 3}

Il fallait corriger la fonction main dans le fichier "mots\_meles.c".

\subsection*{1) Fonction main corrigée}
\begin{lstlisting}[caption=La fonction main corrigée]
#include <stdio.h>
#include "mots_meles.h" /* Erreur : le mauvais fichier .h etait inclus */

int main () {
	int i, j;
	int lg;
	int encore;
	typeCoup coup;
	char grilleJeu[TAILLEGRILLE][TAILLEGRILLE]={ {'A','I','N','T','J'},
												 {'R','F','H','S','E'},
												 {'K','A','H','N','U'},
												 {'P','A','H','O','T'},
												 {'E','S','A','C','Y'} };
	typeMot motsATrouver[LGDICO]={ {2, {'I', 'F'}, 0},
								   {3, {'I', 'N', 'T'}, 0},
								   {4, {'C', 'H', 'A', 'R'}, 0}, /* Erreur : il manquait une virgule */
								   {4, {'C', 'A', 'S', 'E'}, 0},
								   {5, {'C', 'O', 'N', 'S', 'T'}, 0} };
	int grilleTrouve[TAILLEGRILLE][TAILLEGRILLE]; /* Erreur : pas de .. mais [][] */
	char motATrouver[TAILLEGRILLE]; 

	for (i=0; i<TAILLEGRILLE; i++) /* Erreur : on met des ; dans les for, pas des virgules */
		for (j=0; j<TAILLEGRILLE; j++) /* Erreur : pas de ; a la fin de la ligne */
			grilleTrouve[i][j]=0;

	afficheGrille(grilleJeu, grilleTrouve, motsATrouver);
	
	do {
		coup=saisirCoup();
		lg=extraireMot(grilleJeu, coup.x, coup.y, coup.dx, coup.dy, motATrouver, coup.lg);
		i=rechercheMot(motsATrouver, motATrouver, lg);

		if (i<LGDICO)
		{
			printf("\nBravo, ");
			afficheMot(motATrouver, lg);
			printf(" trouve !!!\n\n");
			motTrouve(grilleTrouve, coup.x, coup.y, coup.dx, coup.dy, lg);
			motsATrouver[i].trouve=1;
		}
		else { /* Erreur : il manquait des accolades */
			afficheMot(motATrouver, lg);
			printf(" pas dans la grille\n");
		}

		afficheGrille(grilleJeu, grilleTrouve, motsATrouver);

		if (toutTrouve(motsATrouver)) {
			printf("BRAVO !!! Vous avez trouve tous les mots\n");
			encore=0;
		}
		else { 
			printf("continue ? (oui 1/ non 0) : ");
			scanf("%d", &encore); /* 2 Erreurs : c'est un %d et il manquait un & devant encore */
		}
	} while (encore!=0);
}
\end{lstlisting}

\lstset{
    language=Make,
    basicstyle=\ttfamily\footnotesize,
    aboveskip={1.0\baselineskip},
    belowskip={1.0\baselineskip},
    columns=fixed,
    extendedchars=true,
    breaklines=true,
    tabsize=4,
    prebreak=\raisebox{0ex}[0ex][0ex]{\ensuremath{\hookleftarrow}},
    frame=lines,
    showtabs=false,
    showspaces=false,
    showstringspaces=false,
    keywordstyle=\color[rgb]{0.627,0.126,0.941},
    commentstyle=\color[rgb]{0.133,0.545,0.133},
    stringstyle=\color[rgb]{01,0,0},
    numbers=left,
    numberstyle=\small,
    stepnumber=1,
    numbersep=10pt,
    captionpos=t,
    escapeinside={\%*}{*)}
}


\section*{Remarques}
En corrigeant, pour vérifier mes résultats, j'ai codé complétement le programme de l'IS (qui, normalement, fonctionne sans problème).
Si vous voulez, vous pouvez modifier des fonctions dans la source pour tester vos propres solutions.
Normalement, les fichiers "mots\_meles.h" et "mots\_meles.c" vous sont fournis avec la correction.
Pour compiler, il suffit d'utiliser gcc sur linux :
\begin{lstlisting}[caption=Compiler le programme]
gcc mots_meles.c -o mots_meles
\end{lstlisting}
Puis, pour lancer le programme en console linux :
\begin{lstlisting}[caption=Lancer le programme]
./mots_meles
\end{lstlisting}

N'hésitez pas à me contacter pour des questions/remarques (à l'adresse \href{mailto:elliot.sisteron@insa-rouen.fr}{elliot.sisteron@insa-rouen.fr}).
Bonne révisions et bonne chance pour votre futur examen,\\

Elliot.


\end{document}
