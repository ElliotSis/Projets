%%% Template originaly created by Karol Kozioł (mail@karol-koziol.net) and modified for ShareLaTeX use

\documentclass[a4paper,11pt]{article}

\usepackage[T1]{fontenc}
\usepackage[utf8]{inputenc}
\usepackage[francais]{babel}
\usepackage{verbatim}
\usepackage{graphicx}
\usepackage{xcolor}

\usepackage[defaultmono]{droidmono}

\usepackage{amsmath,amssymb,amsthm,textcomp}
\usepackage{enumerate}
\usepackage{multicol}
\usepackage{tikz}
\usepackage{hyperref}

\usepackage{geometry}
\geometry{total={210mm,297mm},
left=25mm,right=25mm,%
bindingoffset=0mm, top=20mm,bottom=20mm}


%\linespread{1.3}

\newcommand{\linia}{\rule{\linewidth}{0.5pt}}

% my own titles
\makeatletter
\renewcommand{\maketitle}{
\begin{center}
\vspace{2ex}
{\huge \textsc{\@title}}
\vspace{1ex}
\\
\linia\\
\@author \hfill \@date
\vspace{4ex}
\end{center}
}
\makeatother
%%%

% custom footers and headers
\usepackage{fancyhdr}
\pagestyle{fancy}
\lhead{}
\chead{}
\rhead{}
\lfoot{Correction DS langage C}
\cfoot{}
\rfoot{Page \thepage}
\renewcommand{\headrulewidth}{0pt}
\renewcommand{\footrulewidth}{0pt}
%

% code listing settings
\usepackage{listings}
\lstset{
    language=C,
    basicstyle=\ttfamily\footnotesize,
    aboveskip={1.0\baselineskip},
    belowskip={1.0\baselineskip},
    columns=fixed,
    extendedchars=true,
    breaklines=true,
    tabsize=4,
    prebreak=\raisebox{0ex}[0ex][0ex]{\ensuremath{\hookleftarrow}},
    frame=lines,
    showtabs=false,
    showspaces=false,
    showstringspaces=false,
    keywordstyle=\color[rgb]{0.627,0.126,0.941},
    commentstyle=\color[rgb]{0.133,0.545,0.133},
    stringstyle=\color[rgb]{01,0,0},
    numbers=left,
    numberstyle=\small,
    stepnumber=1,
    numbersep=10pt,
    captionpos=t,
    escapeinside={\%*}{*)}
}
\renewcommand\lstlistingname{Code source}
%%%----------%%%----------%%%----------%%%----------%%%

\begin{document}

\title{Correction DS langage C}

\author{Elliot S.}

\date{2013/2014}

\maketitle

\section*{Question 1}

Il était demandé de compléter le fichier "mots\_meles.h".

\subsection*{1) Macro max(a,b)}

\begin{lstlisting}[caption=La macro max]
/* macro pour le maximum de a et b */
#define max(a,b) (((a)<(b))?(b):(a))
\end{lstlisting}

\subsection*{2) Type typeJeu}

\begin{lstlisting}[caption=Le type typeJeu]
/* type pour definir un jeu de mots meles */
typedef struct {
	int tailleGrille; // taille de la grille
	char** grille; // matrice dynamique de caracteres
	int** grilleTrouve; // matrice dynamique de booleens
	int tailleDico; // nombre de mots a trouver
	typeMot* motsATrouver; // tableau dynamique des mots a trouver
} typeJeu;
\end{lstlisting}

\newpage{}

\section*{Question 2}

Il s'agissait ici d'écrire quelques fonctions dans le fichier "mots\_meles.c".

\subsection*{1) Fonction coupValide}
\begin{lstlisting}[caption=La fonction coupValide]
int coupValide(typeCoup coup, int tailleGrille){
	int xi = coup.cinit.x, yi = coup.cinit.y, xf = coup.cfinal.x, yf = coup.cfinal.y;
	int dansGrille = (xi >= 0) && (xi < tailleGrille) && (yi >= 0) && (yi < tailleGrille) && (xf >= 0) && (xf < tailleGrille) && (yf >= 0) && (yf < tailleGrille);
	int memeHorizontale = (yi == yf);
	int memeVerticale = (xi == xf);
	int memeDiagonale = (max(xf - xi, xi - xf) == max(yf - yi, yi - yf));
	return (dansGrille && (memeVerticale || memeHorizontale || memeDiagonale));
}
\end{lstlisting}

\subsection*{2) Fonction saisirCoup}
\begin{lstlisting}[caption=La fonction saisirCoup]
typeCoup saisirCoup(int tailleGrille){
	typeCoup res;
	do{
		printf("Debut du mot (x,y) : ");
		scanf("%d,%d", &(res.cinit.x), &(res.cinit.y));
		printf("fin du mot (x,y) : ");
		scanf("%d,%d", &(res.cfinal.x), &(res.cfinal.y));
	}while(!coupValide(res, tailleGrille));
	return res;
}
\end{lstlisting}

\subsection*{3) Fonction extraireMot}
\begin{lstlisting}[caption=La fonction extraireMot]
char* extraireMot(char **grille, typeCoup coup){
	int i;
	int xi = coup.cinit.x, yi = coup.cinit.y, xf = coup.cfinal.x, yf = coup.cfinal.y;
	int dx = xf - xi, dy = yf - yi;
	int lgRes = max(max(dx, -dx), max(dy, -dy)) + 1;
	char* res = (char*)malloc((lgRes+1)*sizeof(char));

	if (dx != 0)
		dx = ((dx<0)?(-1):1);
	if (dy != 0)
		dy = ((dy<0)?(-1):1);

	for(i = 0 ; i < lgRes ; i++)
		res[i] = grille[xi + i*dx][yi + i*dy];

	res[lgRes] = '\0';

	return res;
}
\end{lstlisting}

\newpage{}
\subsection*{4) Fonction toutTrouve}
\begin{lstlisting}[caption=La fonction toutTrouve]
int toutTrouve(typeMot *motsATrouver, int nbMots){
	int i = 0;
	while(i < nbMots && motsATrouver[i].trouve)
		i++;
	return (i == nbMots);
}
\end{lstlisting}

\section*{Question 3}

Il fallait coder la fonction d'allocation de la matrice de char.

\subsection*{1) Fonction alloueGrilleChar}
\begin{lstlisting}[caption=La fonction alloueGrilleChar]
char ** alloueGrilleChar(int tailleGrille){
	int i;
	char** res = (char**)malloc(tailleGrille*sizeof(char*));
	char* tmp = (char*)malloc(tailleGrille*tailleGrille*sizeof(char));
	if (res == NULL || tmp == NULL)
		return NULL;
	for(i = 0 ; i < tailleGrille ; i++)
		res[i] = &tmp[i*tailleGrille];
	return res;
}
\end{lstlisting}

\section*{Question 4}

Il était demandé de coder la fonction d'affichage du jeu.

\subsection*{1) Fonction afficheJeu}
\begin{lstlisting}[caption=La fonction afficheJeu]
void afficheJeu(typeJeu unJeu){
	int i,j;
	printf("mot meles :\n");
	printf("   X ");
	for (i = 0; i < unJeu.tailleGrille; i++)
		printf("%d  ", i);
	printf("\n");
	for (i = 0; i < unJeu.tailleGrille; i++){
		printf("   %d", i);
		for (j = 0; j < unJeu.tailleGrille; j++)
			if(unJeu.grilleTrouve[i][j])
				printf("(%c)", unJeu.grille[i][j]);
			else
				printf(" %c ", unJeu.grille[i][j]);
		printf("\t\t");
		if(!(unJeu.motsATrouver[i].trouve)){
			j = 0;
			while (unJeu.motsATrouver[i].mot[j]){
				printf("%c", unJeu.motsATrouver[i].mot[j]);
				j++;
			}
		}
		printf("\n");
	}
}
\end{lstlisting}

\section*{Question 5}

Il fallait ici s'occuper de fonctions relatives à la persistance du jeu.

\subsection*{1) Fonction chargeJeu}
\begin{lstlisting}[caption=La fonction chargeJeu]
int chargeJeu(typeJeu *unJeu){
	int trouve;
	FILE* fic = fopen("GRILLE.DATA", "r");
	if (fic != NULL){
		unJeu->grille = lireGrille(fic, &(unJeu->tailleGrille));
		fscanf(fic, "[TROUVE=%d]\n", &trouve);
		unJeu->grilleTrouve = alloueGrilleInt(unJeu->tailleGrille);
		int i,j;
		if (trouve){
			for(i = 0; i < unJeu->tailleGrille ; i++){
				for (j = 0; j < unJeu->tailleGrille ; j++)
						fscanf(fic, "%d ", &(unJeu->grilleTrouve[i][j]));
				fscanf(fic, "\n");
			}
		}
		else{
			for(i = 0; i < unJeu->tailleGrille ; i++)
				for (j = 0; j < unJeu->tailleGrille ; j++)
						unJeu->grilleTrouve[i][j] = 0;
		}
		unJeu->motsATrouver = lireMotsATrouver(fic, unJeu->tailleGrille, &(unJeu->tailleDico));
	}
	else return 0;
	fclose(fic);
	return 1;
}
\end{lstlisting}

\subsection*{2) Fonction lireMotsATrouver}
\begin{lstlisting}[caption=La fonction lireMotsATrouver]
typeMot * lireMotsATrouver(FILE *fic, int tailleGrille, int *tailleDico){
	typeMot* res;
	int i,j;
	fscanf(fic, "[TAILLEDICO=%d]\n", tailleDico);
	res = (typeMot*)malloc((*tailleDico)*sizeof(typeMot));
	for (i = 0 ; i < *tailleDico ; i++){
		res[i].mot = (char*)malloc(tailleGrille*sizeof(char));
		fscanf(fic, "%s %d\n", res[i].mot, &(res[i].trouve));
	}
	return res;
}
\end{lstlisting}

\subsection*{3) Fonction sauveJeu}
\begin{lstlisting}[caption=La fonction sauveJeu]
int sauveJeu(typeJeu unJeu){
	FILE* fic = fopen("GRILLE.DATA", "w");
	if (fic != NULL){
		int i, j, trouve = 0;
		fprintf(fic, "[TAILLEGRILLE=%d]\n", unJeu.tailleGrille);
		for(i = 0 ; i < unJeu.tailleGrille ; i++){
			for(j = 0 ; j < unJeu.tailleGrille ; j++)
				fprintf(fic, "%c ", unJeu.grille[i][j]);
			fprintf(fic, "\n");
		}
		i = 0;
		while (i < unJeu.tailleDico && !trouve){
			trouve = trouve || unJeu.motsATrouver[i].trouve;
			i++;
		}
		fprintf(fic, "[TROUVE=%d]\n", trouve);
		if (trouve){
			for(i = 0; i < unJeu.tailleGrille ; i++){
				for (j = 0; j < unJeu.tailleGrille ; j++)
					fprintf(fic, "%d ", unJeu.grilleTrouve[i][j]);
				fprintf(fic, "\n");
			}
		}
		fprintf(fic, "[TAILLEDICO=%d]\n", unJeu.tailleDico);
		for (i = 0; i < unJeu.tailleDico; i++){
			fprintf(fic, "%s %d\n", unJeu.motsATrouver[i].mot, unJeu.motsATrouver[i].trouve);
		}
	}
	else return 0;
	fclose(fic);
	return 1;
}
\end{lstlisting}

\lstset{
    language=Make,
    basicstyle=\ttfamily\footnotesize,
    aboveskip={1.0\baselineskip},
    belowskip={1.0\baselineskip},
    columns=fixed,
    extendedchars=true,
    breaklines=true,
    tabsize=4,
    prebreak=\raisebox{0ex}[0ex][0ex]{\ensuremath{\hookleftarrow}},
    frame=lines,
    showtabs=false,
    showspaces=false,
    showstringspaces=false,
    keywordstyle=\color[rgb]{0.627,0.126,0.941},
    commentstyle=\color[rgb]{0.133,0.545,0.133},
    stringstyle=\color[rgb]{01,0,0},
    numbers=left,
    numberstyle=\small,
    stepnumber=1,
    numbersep=10pt,
    captionpos=t,
    escapeinside={\%*}{*)}
}


\section*{Remarques}
En corrigeant, pour vérifier mes résultats, j'ai codé complétement le programme du DS (qui, normalement, fonctionne sans problème).
Si vous voulez, vous pouvez modifier des fonctions dans la source pour tester vos propres solutions.
Normalement, les fichiers "mots\_meles.h" et "mots\_meles.c" vous sont fournis avec la correction.
Pour compiler, il suffit d'utiliser gcc sur linux :
\begin{lstlisting}[caption=Compiler le programme]
gcc mots_meles.c -o mots_meles
\end{lstlisting}
Puis, pour lancer le programme en console linux :
\begin{lstlisting}[caption=Lancer le programme]
./mots_meles
\end{lstlisting}

N'hésitez pas à me contacter pour des questions/remarques (à l'adresse \href{mailto:elliot.sisteron@insa-rouen.fr}{elliot.sisteron@insa-rouen.fr}).
Bonne révisions et bonne chance pour votre futur examen,\\

Elliot.


\end{document}
