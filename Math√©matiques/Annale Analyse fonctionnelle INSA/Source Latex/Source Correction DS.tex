%%% Template originaly created by Karol Kozioł (mail@karol-koziol.net) and modified for ShareLaTeX use

\documentclass[a4paper,11pt]{article}

\usepackage[T1]{fontenc}
\usepackage[utf8]{inputenc}
\usepackage[francais]{babel}
\usepackage{verbatim}
\usepackage{graphicx}
\usepackage{xcolor}

\usepackage[defaultmono]{droidmono}

\usepackage{amsmath,amssymb,amsthm,textcomp}
\usepackage{enumerate}
\usepackage{multicol}
\usepackage{tikz}
\usepackage{hyperref}

\usepackage{geometry}
\geometry{total={210mm,297mm},
left=25mm,right=25mm,%
bindingoffset=0mm, top=20mm,bottom=20mm}
\usepackage{tikz,tkz-tab}

%\linespread{1.3}

\newcommand{\linia}{\rule{\linewidth}{0.5pt}}

% my own titles
\makeatletter
\renewcommand{\maketitle}{
\begin{center}
\vspace{2ex}
{\huge \textsc{\@title}}
\vspace{1ex}
\\
\linia\\
\@author \hfill \@date
\vspace{4ex}
\end{center}
}
\makeatother
%%%

% custom footers and headers
\usepackage{fancyhdr}
\pagestyle{fancy}
\lhead{}
\chead{}
\rhead{}
\lfoot{Correction DS Analyse Fonctionnelle}
\cfoot{}
\rfoot{Page \thepage}
\renewcommand{\headrulewidth}{0pt}
\renewcommand{\footrulewidth}{0pt}
%

\theoremstyle{plain}
\newtheorem*{defith}{Définition/Théorème}
\newtheorem*{theo}{Théorème}
\newtheorem*{coro}{Corollaire}

\theoremstyle{definition}
\newtheorem*{defi}{Définition}
\newtheorem*{expl}{Exemple}
\DeclareMathOperator{\arginf}{arginf}
\renewcommand{\Re}{\operatorname{\mathfrak{Re}}}

\begin{document}
\renewcommand{\proofname}{Preuve}

\title{Correction DS Analyse Fonctionnelle}

\author{Elliot S.}

\date{2013/2014}

\maketitle

\section*{Partie 1}
\subsection*{Question 1 : Espaces compacts}
Il était possible de donner plusieurs définitions selon l'espace dans lequel on travaille.
La définition la plus générale est celle de Borel-Lebesgue :
\begin{defi} (Recouvrement)
Soit $E$ un ensemble quelconque et $A$ une partie de $E$. Soit $(U_i)_{i \in I}$ une famille de parties de $E$.
On dira que $(U_i)_{i \in I}$ recouvre $A$ si et seulement si $A \subset \bigcup_{i \in I} U_i$.
\end{defi}

\begin{defi} (Compact au sens de Borel-Lebesgue)
Soit $T$ un espace topologique (i.e. un espace dans lequel on peut parler d'ouverts et de fermés). Soit $X$ une partie de $T$.
On dit que $X$ est compact dans $T$ si et seulement si chaque fois qu'il est recouvert par des ouverts de $T$, on peut le recouvrir par un nombre fini d'entre eux.
\end{defi}

Dans un espace métrique (qui est un cas particulier d'espace topologique), on a une autre définition qui est équivalente :

\begin{defith} (Compact au sens de Bolzano-Weierstrass)
Soit $(M, d)$ un espace métrique. Soit $X$ une partie de $(M, d)$.
Alors, $X$ est compact dans $M$ si est seulement si toute suite de $X$ admet une valeur d'adhérence dans $X$ (i.e. on peut trouver une suite extraite convergente vers un élément de $X$).
\end{defith}

Enfin, lorsque l'on se restreint uniquement à un espace vectoriel réel ou complexe de dimension finie, on a la caractérisation suivante :
\begin{defith} (Caractérisation des compacts dans un espace vectoriel de dimension finie)
Soit $V$ un espace vectoriel réel ou complexe de dimension finie.
Soit $X$ une partie de $V$.
Alors, $X$ est compact dans $V$ si et seulement si $X$ est un fermé borné.
\end{defith}

On avait deux exemples à traiter :
\begin{expl}
$X_{1} = \left\{1, \frac{1}{2}, \frac{2}{3}, \frac{3}{4}, ... , \frac{n}{n+1}, ...\right\}$ est-il compact ?
\end{expl}
$X_{1}$ est une partie de $\mathbb{R}$ qui est un espace vectoriel de dimension finie. 
Cela revient donc à se demander si $X_{1}$ est un fermé borné.
C'est clairement un borné car $X_{1} \subset \left[\frac{1}{2}, 1\right]$.
On remarque que le complémentaire de $X_{1}$ s'exprime de la manière suivante : 
\[
	X_{1}^{c} = \left]-\infty, \frac{1}{2}\right[\cup \left(\bigcup_{n \in \mathbb{N}^*} \left]\frac{n}{n+1}, \frac{n+1}{n+2}\right[\right) \cup \left]1, +\infty\right[
\]
C'est une réunion infinie d'ouverts, donc c'est un ouvert.
Ainsi, comme son complémentaire est un ouvert, $X_{1}$ est un fermé et donc un compact.

\begin{expl}
$X_{2} = \left\{1, \frac{1}{2}, \frac{1}{3}, \frac{1}{4}, ... , \frac{1}{n}, ...\right\}$ est-il compact ?
\end{expl}
Non, car il n'est pas fermé.
En effet, on a $\left(\frac{1}{n}\right)_{n \in \mathbb{N}^*}$ qui est une suite de $X_{2}$ mais qui converge vers la valeur $0$ qui n'est pas un élément de $X_{2}$.

\subsection*{Question 2 : Opérateur continu}
Commençons par définir cette notion. Soient $\left(E, ||\cdot||_{E}\right)$ et $\left(F, ||\cdot||_{F}\right)$ deux espaces vectoriels normés de corps $\mathbb{K}$ :
\begin{defi} (Application linéaire)
Une application $L : E \to F$ est dite linéaire si et seulement si $\forall \lambda \in \mathbb{K}, \ \forall \left(x,y\right) \in E^2, \ L\left(\lambda x + y\right) = \lambda L\left(x\right) + L\left(y\right)$.
\end{defi}

\begin{defi} (Application continue dans des espaces vectoriels normés)
Une application $C : E \to F$ est dite continue sur $E$ si et seulement si :
\[
\forall x \in E, \ \forall \varepsilon > 0, \ \exists \delta > 0 \ / \ \forall y \in E, \ \left(||x-y||_{E} \leq \delta \Rightarrow ||C(x) - C(y)||_{F} \leq \varepsilon \right)
\]
On dira que $C$ est uniformément continue si et seulement si :
\[
\forall \varepsilon > 0, \ \exists \delta > 0 \ / \ \forall x,y \in E, \ \left(||x-y||_{E} \leq \delta \Rightarrow ||C(x) - C(y)||_{F} \leq \varepsilon \right)
\]
\end{defi}

\begin{defi} (Opérateur continu)
On appelle opérateur toute application entre deux espaces vectoriels topoplogiques.
En particulier, si cette application est linéaire ou continue (ou les deux), on qualifiera l'opérateur de linéaire ou continu (ou les deux).
\end{defi}

On alors un théorème pour caractériser la continuité des applications linéaires :
\begin{theo} (Caractérisation de la continuité des applications linéaires)
Soit $O : E \to F$ un opérateur linéaire.
On a équivalence entre ces différentes propositions :
\begin{enumerate}
	\item $O$ est continu en $0$
	\item $O$ est continu
	\item $O$ est uniformément continu
	\item $O$ est lipschitzien
	\item $O$ est borné sur la boule unité fermée, i.e. $\exists M > 0 \ / \ \forall x \in E, \ ||x||_E \leq 1 \Rightarrow ||O(x)||_{F} \leq M$
	\item $O$ est borné, i.e. $\exists M > 0 \ / \ \forall x \in E, \ ||O(x)||_{F} \leq M||x||_{E}$
\end{enumerate}
\end{theo}
On devait alors démontrer que la deuxième proposition est équivalente à la dernière :
\begin{proof}
	\begin{itemize}
		\item Supposons que $O$ est continu. Alors $O$ est continu en $0$ et donc :
		\[
			\forall \varepsilon > 0, \ \exists \delta > 0 \ / \ \forall y \in E, \ \left(||y||_{E} \leq \delta \Rightarrow ||O(y)||_{F} \leq \varepsilon \right)
		\]
		Choisissons par exemple $\varepsilon = 1$.
		Soit $x \in E$ tel que $||x||_{E} \leq 1$.
		Alors $||\delta x|| \leq \delta$.
		Posons $y = \delta x$.
		Dès lors : 
		\[
			||x||_{E} \leq 1 \Rightarrow ||y||_{E} \leq \delta \Rightarrow ||O(y)||_{F} \leq \varepsilon = 1 \Rightarrow ||O(x)||_{F} \leq \frac{1}{\delta}
		\]
		Posons $M = \frac{1}{\delta}$.
		Ainsi $O$ est borné par $M$ sur la boule unité fermée, en effet :
		\[
			\forall x \in E, \ ||x||_{E} \leq 1 \Rightarrow ||O(x)||_{F} \leq M
		\]	

		Soit $x \in E$, si $x = 0_{E}$ la proposition 6 est vérifiée, supposons donc $x \neq 0_{E}$.
		On a donc :
		\[
			||\frac{x}{||x||_{E}}||_{E} = 1 \leq 1 \Rightarrow ||O\left(\frac{x}{||x||_{E}}\right)||_{F} \leq M \Rightarrow ||O(x)||_{F} \leq M||x||_{E}
		\]
		\item Supposons maintenant que $O$ est borné, i.e. :
		\[
			\exists M > 0 \ / \ \forall x \in E, \ ||O(x)||_{F} \leq M||x||_{E}
		\]
		Soit $\varepsilon > 0$, posons $\delta = \frac{\varepsilon}{M}$, alors :
		\[
			\forall x \in E, \ ||x||_{E} \leq \delta = \frac{\varepsilon}{M} \Rightarrow ||O(x)||_{F} \leq M||x||_{E} \leq \varepsilon
		\]
		Donc $O$ est continu en $0$.
		En particulier, on a donc :
		\[
			\forall x,y \in E, \ ||x - y||_{E} \leq \delta \Rightarrow ||O(x -y)||_{F} = ||O(x) - O(y)||_{F} \leq \varepsilon
		\]
		D'où la continuité (et même la continuité uniforme).
	\end{itemize}
\end{proof}

On avait l'exemple suivant à étudier :

\begin{expl}
L'opérateur :
\[\begin{array}{ccccc}
O & : & \mathcal{C}^{\infty}\left([0,1] \to \mathbb{R}\right) &\longrightarrow  & \mathcal{C}^{\infty}\left([0,1] \to \mathbb{R}\right)\\
	& & f &\longmapsto& f'\\
\end{array}\] est-il continu ?
\end{expl}
Et bien, la réponse est… oui et non !
En effet, le sujet ne donnant pas la norme utilisée, la réponse varie selon celle que l'on prend.
Il fallait d'abord montrer que l'opérateur est linéaire, ce qui est évident par la linéarité de la dérivée.
\begin{itemize}
\item Pour la norme infinie, par exemple, l'opérateur n'est pas continu.
\[\begin{array}{ccccc}
O & : & \left(\mathcal{C}^{\infty}\left([0,1] \to \mathbb{R}\right), ||\cdot||_{\infty}\right) &\longrightarrow  & \left(\mathcal{C}^{\infty}\left([0,1] \to \mathbb{R}\right), ||\cdot||_{\infty}\right)\\
	& & f &\longmapsto& f'\\
\end{array}\] 
En effet, posons $(f_n)_{n \in \mathbb{N}}$ la suite de fonctions définie pour tout $n \in \mathbb{N}$ et pour tout $x \in [0,1]$ par :
\[
	f_{n}(x) = \sin(nx)
\]
D'où :
\[
	f'_{n}(x) = n\cos(nx)
\]
Ainsi, $||f_{n}||_{\infty} = 1 \leq 1$ pour tout $n$, mais $||O(f_{n})||_{\infty} = ||f'_{n}||_{\infty} = n$.
En faisant tendre $n$ vers l'infini, on se rend compte que $O$ n'est pas borné sur la boule unité fermée et donc $O$ n'est pas continu pour la norme infinie.
\item Pour la norme suivante :
\[
	||f|| = ||f||_{\infty} + ||f'||_{\infty}
\]
L'opérateur 
\[\begin{array}{ccccc}
O & : & \left(\mathcal{C}^{\infty}\left([0,1] \to \mathbb{R}\right), ||\cdot||\right) &\longrightarrow  & \left(\mathcal{C}^{\infty}\left([0,1] \to \mathbb{R}\right), ||\cdot||_{\infty}\right)\\
	& & f &\longmapsto& f'\\
\end{array}\] 
est continu.\\
En effet, 
\[
\forall f \in \mathcal{C}^{\infty}\left([0,1] \to \mathbb{R}\right), \ ||O(f)||_{\infty} = ||f'||_{\infty} \leq ||f|| = ||f||_{\infty} + ||f'||_{\infty}
\]
Ainsi, $O$ est borné (par la constante $1$) et donc il est continu.
\end{itemize}

\subsection*{Question 3 : Espace de Hilbert}
Donnons la définition de cet espace :
\begin{defi} (Espace pré-hilbertien)
Soit $E$ un espace vectoriel de corps $\mathbb{K} = \mathbb{R}$ ou $\mathbb{C}$.
Soit $\left<\cdot,\cdot\right> \ : E^2 \to \mathbb{K}$ une forme vérifiant :
\begin{enumerate}
	\item $\left<\cdot,\cdot\right>$ est linéaire à gauche, i.e. $\forall \lambda \in \mathbb{K}, \ \forall (x,y,z) \in E^3, \ \left<\lambda x + y,z\right> = \lambda\left<x,z\right> + \left<y,z\right>$;
	\item $\left<\cdot,\cdot\right>$ est symétrique hermitienne, i.e. $\forall (x,y) \in E^2, \ \left<x,y\right> = \overline{\left<y,x\right>}$;
	\item $\left<\cdot,\cdot\right>$ est définie, i.e. $\forall x \in E, \ \left<x,x\right> = 0 \Leftrightarrow x = 0_{E}$;
	\item $\left<\cdot,\cdot\right>$ est positive, i.e. $\forall x \in E, \ \left<x,x\right> \ \geq 0$.
\end{enumerate}
On dira alors que $\left<\cdot,\cdot\right>$ est un produit scalaire hermitien et on qualifiera l'espace $(E, \left<\cdot,\cdot\right>)$ de pré-hilbertien.
\end{defi}

\begin{defi} (Espace complet)
Soit $(E,d)$ un espace métrique.
On dira que $E$ est complet si et seulement si toute suite de Cauchy $(x_{i})_{i \in \mathbb{N}}$ de $E$ admet une limite $x \in E$ pour la distance $d$.
En particulier, si $E$ est un espace vectoriel normé, on le qualifiera d'espace de Banach.
\end{defi}

\begin{defi} (Espace de Hilbert)
Soit $(H, \left<\cdot,\cdot\right>)$ un espace pré-hilbertien.
Si $H$ est aussi de Banach avec la norme induite par son produit scalaire $||\cdot|| = \sqrt{\left<\cdot,\cdot\right>}$, alors on dira que $H$ est un espace de Hilbert.
\end{defi}

Deux exemples étaient à traiter :
\begin{expl}
$H_{1} = \left(\mathcal{C}\left([0,\frac{\pi}{2}] \to \mathbb{R}\right), ||\cdot||_{\infty}\right)$ est-il de Hilbert ?
\end{expl}

On sait que tout espace de la forme $\left(\mathcal{C}\left([a,b] \to \mathbb{R}\right), ||\cdot||_{\infty}\right)$ est de Banach.
Donc, $H_{1}$ est de Banach.
Est-il pré-hilbertien ?
On a le théorème du cours suivant :
\begin{theo} (Fréchet-von Neumann-Jordan)
Soit $(E, ||\cdot||)$ un espace vectoriel normé.
La norme $||\cdot||$ découle d'un produit scalaire (i.e. elle s'exprime sous la forme $||\cdot|| = \sqrt{\left<\cdot,\cdot\right>}$) si et seulement si elle vérifie la \og règle du parallélogramme \fg{} :
\[
	\forall x,y \in E^2, \ 2||x||^2 + 2||y||^2 = ||x + y||^2 + ||x -y||^2
\]
De plus, ce produit scalaire est unique.
\end{theo}
Ainsi $H_{1}$ est pré-hilbertien si et seulement si la norme infinie vérifie la règle du parallélogramme.
Trouvons un contre-exemple, soient $f$ et $g$ deux fonctions telles que, $\forall x \in \left[0, \frac{\pi}{2}\right]$:
\[\begin{aligned}
f(x) = \cos(x) + \sin(x)\\
g(x) = \cos(x) - \sin(x)
\end{aligned}
\]
On remarque que $f' = g$ et que $g' = -f$.
On cherche le maximum de $f$,
\[
\begin{aligned}
	f'(x) = 0 & \Leftrightarrow \cos(x) = \sin(x)\\
			& \Leftrightarrow x = \frac{\pi}{4}
\end{aligned}
\]
D'où le tableau de variation suivant :
\[
\begin{tikzpicture}
\tkzTabInit{$x$/1,$f'(x)$/1,$f(x)$/2}{$0$,$\frac{\pi}{4}$,$\frac{\pi}{2}$}
\tkzTabLine{,+,z,-}
\tkzTabVar{-/$1$,+/$\sqrt{2}$,-/$1$}
\end{tikzpicture}
\]
Par le même raisonnement, pour $g$ on a :
\[
\begin{tikzpicture}
\tkzTabInit{$x$/1,$g'(x)$/1,$g(x)$/2}{$0$,$\frac{\pi}{2}$}
\tkzTabLine{,-,}
\tkzTabVar{+/$1$,-/$-1$}
\end{tikzpicture}
\]
Ainsi, 
\[
2 \left(||f||_{\infty}^2 + ||g||_{\infty}^2\right) = 6
\]
Et,
\[
||f + g||_{\infty}^2 + ||f -g||_{\infty}^2 = 4 + 4 = 8
\]
Donc la règle du parallélogramme n'est pas vérifiée : $H_{1}$ n'est pas pré-hilbertien, donc pas de Hilbert.
\begin{expl}
$H_{2} = \left(l^2\left(\mathbb{K}\right), ||\cdot||_{2}\right)^{'} = \mathcal{L}_{c}\left(l^2\left(\mathbb{K}\right), \mathbb{K}\right)$ (l'espace dual topologique de $l^2\left(\mathbb{K}\right)$, i.e. l'ensemble des formes linéaires continues de $l^2\left(\mathbb{K}\right)$) est-il de Hilbert ?
\end{expl}

On sait, d'après le cours et les TD, que l'espace $\left(l^2\left(\mathbb{K}\right), ||\cdot||_{2}\right)$ est de Hilbert, donc de Banach.
Et, $\mathbb{K}$ est clairement de Banach.
Or,
\begin{theo}
Soient $E$ et $F$ deux espaces de Banach.
Alors $\mathcal{L}_{c}(E,F)$ est lui aussi de Banach.
\end{theo}
Ainsi, $\mathcal{L}_{c}\left(l^2\left(\mathbb{K}\right), \mathbb{K}\right)$ est de Banach.
De plus, l'espace dual d'un espace de Hilbert est muni d'une norme dérivant du même produit scalaire hermitien (cf. théorème de Riesz-Fréchet).
Ainsi, $H_2$ est de Hilbert.

\subsection*{Question 4 : Théorème de Riesz-Fréchet}
Rappelons le théorème de Riesz-Fréchet :
\begin{theo} (Riesz-Fréchet)
Soit $H$ un espace de Hilbert de corps $\mathbb{K}$.
Soit $f$ une application linéaire continue de $H$ dans $\mathbb{K}$, autrement dit $f$ est un élément du dual topologique de $H$ : $f \in H^{'} = \mathcal{L}_{c} \left(H,\mathbb{K}\right)$.
Alors il existe un unique $y \in H$ tel que 
\[
	\forall x \in H, \ f(x) = \left<y,x\right>
\]
De plus, $|||f||| = ||y||$.
\end{theo}

Il fallait alors démontrer l'unicité :
\begin{proof}
Soit $f \in H^*$.
Supposons que l'on ait $y_1$ et $y_2$ deux éléments distincts de $H$ tels que $\forall x \in H,$
\[\begin{aligned}
 f(x) = \left<y_1,x\right>\\
 f(x) = \left<y_2,x\right>
\end{aligned}
\]
Ainsi, 
\[
	\forall x \in H, \ \left<y_1-y_2,x\right> = 0
\]
En particulier pour $x = y_1 - y_2$, 
\[
	||y_1 - y_2||^2 = 0
\]
Par définition de la norme on a donc $y_1 = y_2$.
C'est absurde par supposition, d'où l'unicité.
\end{proof}

\subsection*{Question 5 : Opérateur compact}
Donnons la définition :
\begin{defi} (Partie relativement compacte)
Soit $T$ un espace topologique.
Soit $A \subset T$.
On dira que $A$ est relativement compacte si et seulement si $\overline{A}$ est compact (i.e. son adhérence est compacte).
\end{defi}

\begin{defi} (Opérateur compact)
Soit $H$ un espace de Hilbert.
Soit $O \in \mathcal{L}(H)$.
On dira que $O$ est compact si et seulement si $O\left(\mathcal{B}_f(0_H,1)\right)$ est relativement compact (i.e. si l'image de la boule unité fermée par $O$ est relativement compacte).
\end{defi}

Il fallait alors décrire tous les opérateurs compacts en dimension finie.
Procédons par analyse-synthèse :
\begin{itemize}
\item \emph{Analyse} : Prenons $H$ un espace de Hilbert de dimension finie, $O$ un opérateur linéaire de $H$ dans $H$.
Pour que $O$ soit compact, il faut que l'adhérence de l'image de la boule unité fermée soit un fermé borné (car on est en dimension finie).
Comme l'adhérence est un fermé, on doit juste trouver une condition pour que $\overline{O\left(\mathcal{B}_f(0_H,1)\right)}$ soit borné.
Or, $O\left(\mathcal{B}_f(0_H,1)\right)$ est borné (et donc son adhérence aussi) si et seulement si $O$ est continu.
On a donc vu que dès qu'un opérateur est compact, il doit être continu : c'est la condition recherchée.
\item \emph{Synthèse} : Prenons maintenant $O$ un opérateur linéaire continu de $H$. Comme il est continu, l'image de la boule unité fermé par $O$ est bornée (donc son adhérence aussi). Or, l'adhérence est un fermé. Donc $\overline{O\left(\mathcal{B}_f(0_H,1)\right)}$ est un fermé borné en dimension finie. Ainsi, c'est un compact. Finalement, $O$ est bien compact.
\end{itemize}
Ce qui démontre que, en dimension finie, $O$ compact $\Leftrightarrow$ $O$ linéaire continu.

\section*{Partie 2}
\subsection*{Question 1 : Continuité et norme d'opérateurs}
Soit $\left(E,||\cdot||_E\right)$ et $\left(F,||\cdot||_F\right)$ deux espaces vectoriels normés.
Soit $O \in \mathcal{L}_c\left(E,F\right)$.
\begin{defith} (Norme d'opérateur continu)
L'application définie par :
\[\begin{aligned}
	|||O||| & = \sup_{||x||_E \neq 0} \frac{||O(x)||_F}{||x||_E} \\
			& = \sup_{||x||_E \leq 1} ||O(x)||_F\\
			& = \sup_{||x||_E = 1} ||O(x)||_F
\end{aligned}
\]
est une norme et elle est appelée norme de l'opérateur $O$, subordonnée à $||\cdot||_E$ et $||\cdot||_F$.
\end{defith}
On rapelle que pour calculer la norme d'un opérateur, on utilise la méthode suivante :
\begin{enumerate}
\item On essaie de trouver une constante $M$ par lequel l'opérateur est borné :
\[
	\forall x \in E, \ ||O(x)||_F \leq M||x||_E
\]
D'où :
\[
	\frac{||O(x)||_F}{||x||_E} \leq M
\]
On en déduit alors que $|||O||| \leq M$ car $|||O|||$ est le plus petit majorant de cette expression et $M$ est un majorant.
Si $M$ est correctement choisie, on peut alors montrer que $|||O||| = M$.
\item On a deux cas de figure : 
\begin{itemize}
\item Soit le sup est atteint, auquel cas on doit trouver un certain vecteur $u$ pour lequel le sup est atteint.
Par exemple, tel que $||u||_E \leq 1$ et $||O(u)||_F = M$.
Ou encore, tel que $||u||_E \neq 0$ et :
\[
	\frac{||O(u)||_F}{||u||_E} = M
\]
\item Soit le sup n'est pas atteint, et là il nous faut trouver une suite (ou une famille) de vecteurs $u_n$ qui converge vers le sup.
Par exemple, telle que à partir d'un certain rang $||u_n||_E \leq 1$ et :
\[
	\lim_{n \to +\infty} ||O(u_n)||_F = M
\]
Ou encore, telle que à partir d'un certain rang $||u_n||_E \neq 0$ et :
\[
	\lim_{n \to +\infty} \frac{||O(u_n)||_F}{||u_n||_E} = M
\]
\end{itemize}
Ce qui montre que $|||O||| = M$.
\end{enumerate}

Deux exemples étaient donnés :
\begin{expl}
Soit $p \in [1, +\infty]$.
Démontrer que l'opérateur :
\[\begin{array}{ccccc}
O_1 & : & \left(l^p\left(\mathbb{K}\right), ||\cdot||_p\right) &\longrightarrow  & \left(l^p\left(\mathbb{K}\right), ||\cdot||_p\right)\\
	& & (x_1,x_2, ...) &\longmapsto& (x_1, x_1,x_2, ...)\\
\end{array}\] 
est continu, calculer sa norme.
\end{expl}

On doit tout d'abord se demander si cet opérateur est linéaire.
Soit $\lambda \in \mathbb{K}$, et $x, y \in l^p\left(\mathbb{K}\right)$.
\[
\begin{aligned}
O_1(\lambda x + y) & = (\lambda x_1 + y_1, \lambda x_1 + y_1, \lambda x_2 + y_2, ...) \\
				   & = \lambda (x_1, x_1, x_2, ...) + (y_1, y_1, y_2, ...) \\
				   & = \lambda O_1(x) + O_1(y)
\end{aligned}
\]
Ainsi $O_1$ est bien linéaire.

On va distinguer deux cas : le cas où $p \in [1, +\infty[$ et le cas où $p = +\infty$.
\begin{itemize}
\item $p \in [1, +\infty[$ :
\begin{itemize}
\item \emph{Continuité} : Soit $x = (x_1, x_2, ...) \in l^p\left(\mathbb{K}\right)$.
Alors,
\[
||O_1(x)||_p = \left(|x_1|^p + \sum_{i = 1}^{\infty}|x_i|^p\right)^\frac{1}{p} \leq \left(2\sum_{i = 1}^{\infty}|x_i|^p\right)^\frac{1}{p} \leq 2^{\frac{1}{p}}||x||_p
\]
Ainsi, $O_1$ est borné, donc $O_1$ est continu.
\item \emph{Norme} : On a vu que, $\forall x \in l^p\left(\mathbb{K}\right), \ ||O_1(x)||_p \leq 2||x||_p$.
D'où $|||O_1||| \leq 2$.
Pour montrer que $|||O_1||| = 2$, il nous suffit donc de trouver un certain $u$ pour lequel la valeur $2$ est atteinte.
Posons $u = (1, 0, 0, ...)$, alors on a bien $||u||_p = 1 \leq 1$ et $||O_1(u)||_p = \left(1 + 1\right)^{\frac{1}{p}} = 2^{\frac{1}{p}}$.
Ainsi, $|||O_1||| = 2^{\frac{1}{p}}$.
\end{itemize}
\item $p = +\infty$ :
\begin{itemize}
\item \emph{Continuité} : Soit $x = (x_1, x_2, ...) \in l^\infty\left(\mathbb{K}\right)$.
Alors,
\[
||O_1(x)||_\infty = ||x||_\infty \leq ||x||_\infty
\]
Ainsi, $O_1$ est borné, donc $O_1$ est continu.
\item \emph{Norme} : On a vu que, $\forall x \in l^p\left(\mathbb{K}\right), \ ||O_1(x)||_\infty = ||x||_\infty$.
Il suffit de prendre n'importe quel vecteur $u$ tel que $||u||_\infty = 1 \leq 1$ et on a immédiatement $||O_1(u)||_\infty = ||u||_\infty = 1$.
Par exemple, $u = (1, 0, 0, ...)$… Donc $|||O_1||| = 1$.
\end{itemize}
\end{itemize}

Passons à l'autre exemple :
\begin{expl}
Soit $k \in \mathcal{C}\left([0,1] \to \mathbb{R}\right)$.
Démontrer que l'opérateur :
\[\begin{array}{ccccc}
O_2 & : & \mathcal{C}\left([0,1] \to \mathbb{R}\right) &\longrightarrow  &  \mathbb{R}\\
	& & f &\longmapsto& \int_0^1 k(t)f(t)dt\\
\end{array}\] 
est continu, calculer sa norme.
\end{expl}
Cet opérateur est clairement linéaire par linéarité de l'intégrale.
Là encore on ne nous donne pas la norme utilisée… On va donc prendre la norme \og usuelle \fg{} dans cet espace qui est la norme infinie.

\begin{itemize}
\item \emph{Continuité} : Soit $f \in \mathcal{C}\left([0,1] \to \mathbb{R}\right)$.
Alors, en utilisant l'inégalité triangulaire :
\[
\begin{aligned}
	|O_2(f)| = \left|\int_0^1 k(t)f(t)dt\right| & \leq \int_0^1 \left|k(t)f(t)\right|dt\\
	& \leq \int_0^1 |k(t)|\sup_{x \in [0,1]}|f(x)|dt\\
	& = \left(\int_0^1 |k(t)|dt\right) ||f||_\infty\\
	& = ||k||_1 ||f||_\infty
\end{aligned}
\]
Ainsi $O_2$ est borné par $||k||_1$ et donc $O_2$ est continu.
\item \emph{Norme} : On a donc $|||O_2||| \leq ||k||_1$.
On a vu tout a l'heure qu'il fallait trouver une fonction atteignant le sup ou une suite de fonctions tendant vers le sup.
Nous allons considérer la famille de fonctions continues $\left(u_\varepsilon\right)_{\varepsilon > 0}$ définie par, $\forall x \in [0,1]$ :
\[
 u_\varepsilon(x) = \left\{ 
  \begin{array}{l l l}
    -1 & \quad \text{si $k(x) \leq -\varepsilon$}\\
    \frac{k(x)}{\varepsilon} & \quad \text{si $k(x) \in ]-\varepsilon,\varepsilon[$}\\
    1 & \quad \text{si $k(x) \geq \varepsilon$}
  \end{array} \right.
\]
On a, $\forall \varepsilon > 0, \ ||u_\varepsilon||_\infty = 1$.
Soit $x \in [0,1]$.
On remarque que $k(x)u_\varepsilon(x) \geq 0$ donc que $O_2(u_\varepsilon) \geq 0$.
Considérons l'expression $|k(x)| - k(x)u_\varepsilon(x)$.
On va essayer de majorer cette expression.
Deux cas se présentent :
\begin{itemize}
\item Si $|k(x)| < \varepsilon$, alors :
\[
	0 \leq |k(x)| - k(x)u_\varepsilon(x) \leq \varepsilon - \frac{k^2(x)}{\varepsilon} \leq \varepsilon
\]
\item Si $|k(x)| \geq \varepsilon$, alors :
\[
	|k(x)| - k(x)u_\varepsilon(x) = |k(x)| - |k(x)| = 0 \leq \varepsilon
\]
\end{itemize}
Ainsi :
\[
\begin{aligned}
	0 \leq |k(x)| - k(x)u_\varepsilon(x) \leq \varepsilon & \Leftrightarrow 0 \leq \int_0^1 |k(t)| - k(t)u_\varepsilon(t)dt \leq \varepsilon\\
	& \Leftrightarrow 0 \leq ||k||_1 - O_2(u_\varepsilon) \leq \varepsilon\\
	& \Leftrightarrow ||k||_1 \geq O_2(u_\varepsilon) \geq ||k||_1 - \varepsilon \\
\end{aligned}
\]
Finalement :
\[
 \lim_{\varepsilon \to 0} |O_2(u_\varepsilon)| = ||k||_1
\]
Ainsi, en posant par exemple $\varepsilon_n = \frac{1}{n}$, on peut créer une suite $u_n = u_{\varepsilon_n}$ qui vérifie $\forall n \in \mathbb{N}, \ ||u_n||_\infty = 1 \leq 1$ et :
\[
 \lim_{n \to +\infty} |O_2(u_n)| = ||k||_1
\]
On peut enfin conclure que $|||O_2||| = ||k||_1$.
\end{itemize}

%Donc selon la norme choisie, cela peut être plus ou moins difficile.
%Pour l'espace d'arrivée, qui est $\mathbb{R}$, ce n'est pas génant car la norme est toujours la valeur absolue (à une constante près).
%Il fallait donc choisir la norme de l'espace de départ.
%Ce n'est pas la seule façon de procéder, mais j'ai choisi la norme $2$, définie par :
%\[
%	\forall f \in \mathcal{C}\left([0,1] \to \mathbb{R}\right), \ ||f||_2 = \left(\int_0^1 \left|f(x)\right|^2\right)^\frac{1}{2}
%\]
%\begin{itemize}
%\item \emph{Continuité} : Soit $f \in \mathcal{C}\left([0,1] \to \mathbb{R}\right)$.
%Alors :
%\[
%	|O_2(f)| = \left|\int_0^1 k(x)f(x)dx\right|
%\]
%On remarque que $\left(\mathcal{C}\left([0,1] \to \mathbb{R}\right),||\cdot||_2\right)$ est pré-hilbertien avec le produit scalaire définit %par, $\forall f,g \in \mathcal{C}\left([0,1] \to \mathbb{R}\right)$ :
%\[
%	\left<f,g\right> = \int_0^1 f(x)g(x)dx 
%\]
%On utilise maintenant l'inégalité de Cauchy-Schwarz :
%\[
%	|O_2(f)| = \left|\left<k,f\right>\right| \leq ||k||_2 ||f||_2
%\]
%Ainsi, $O_2$ est borné, donc $O_2$ est continu.
%\item \emph{Norme} : On a donc $|||O_2||| \leq ||k||_2$.
%On a vu tout a l'heure qu'il fallait trouver une fonction atteignant le sup.
%Si $k$ est la fonction nulle, alors clairement $|||O_2||| = 0$ et donc $|||O_2||| = ||k||_2$.
%Supposons maintenant $k$ non identiquement nulle.
%Alors, $||k||_2 \neq 0$.
%Prenons la fonction $u$ définie par $\forall x \in [0,1], \ u(x) = k(x)$.
%Alors, 
%\[
%	\frac{|O_2(u)|}{||u||_2} = \frac{||k||_2^2}{||k||_2} = ||k||_2
%\]
%Ainsi, le sup est atteint en $u$.
%D'où $|||O_2||| = ||k||_2$.
%\end{itemize}

\subsection*{Question 2 : Opérateur auto-adjoint}
Rappelons ce qu'est un opérateur auto-adjoint :
\begin{defi} (Opérateur adjoint)
Soient $H_1$ et $H_2$ deux espaces de Hilbert.
Soit $O : H_1 \to H_2$ un opérateur linéaire continu.
On dira que l'opérateur $O^* : H_2 \to H_1$ est adjoint à $O$ si et seulement si :
\[
	\forall x \in H_1, \ \forall y \in H_2, \ \left<O(x),y\right>_{H_2} = \left<x,O^*(y)\right>_{H_1} 
\]
\end{defi}

\begin{defi} (Opérateur auto-adjoint)
Soit $H$ un espace de Hilbert.
Soit $O \in \mathcal{L}_c(H)$.
Si $O^* = O$, alors on dira que $O$ est auto-adjoint.
\end{defi}

On devait alors démontrer le théorème suivant :
\begin{theo} (Caractérisation des opérateurs auto-adjoints)
Soit $H$ un espace de Hilbert.
Soit $O$ un opérateur linéaire continu de $H$ dans $H$.
Alors :
\[
	O^* = O \Leftrightarrow \left<O(x),x\right> \in \mathbb{R}
\]
\end{theo}

\begin{proof}
Soit $H$ un espace de Hilbert.
Soit $O \in \mathcal{L}_c(H)$.
\begin{itemize}
\item \emph{$\Rightarrow$} : Supposons que $O$ est auto-adjoint.
Alors, 
\[
	\forall x, y \in H, \ \left<O(x),y\right> = \left<x,O(y)\right>
\]
En particulier,
\[
	\forall x \in H, \ \left<O(x),x\right> = \left<x,O(x)\right>
\]
D'où, par symétrie hermitienne :
\[
	\left<O(x),x\right> = \left<x,O(x)\right> = \overline{\left<O(x),x\right>}
\]
En posant $z = \left<O(x),x\right>$, on a $z = \overline{z}$.
Ainsi, $z \in \mathbb{R}$.
Finalement, on a bien :
\[
	\left<O(x),x\right> \in \mathbb{R}
\]
\item \emph{$\Leftarrow $} : Supposons maintenant que $\forall x \in H, \ \left<O(x),x\right> \in \mathbb{R}$. On a donc $\left<O(x),x\right> = \overline{\left<O(x),x\right>} = \left<x,O(x)\right>$.
Soit $x, y \in H$.
\[
\begin{aligned}
	4\left<O(x),y\right> & = \left<O(x + y), x + y\right> - \left<O(x - y), x - y\right> + i\left<O(x + iy), x + iy\right> -i\left<O(x - iy), x - iy\right>\\
	& = \left<x + y, O(x + y)\right> - \left<x - y, O(x - y)\right> + i\left<x + iy, O(x + iy)\right> -i\left<x - iy, O(x - iy)\right>\\
	& = 4\left<x,O(y)\right>
\end{aligned}
\]
Ainsi $\left<O(x),y\right> = \left<x,O(y)\right>$ et donc $O$ est bien auto-adjoint.
\end{itemize}

\end{proof}

\subsection*{Question 3 : Projections orthogonales sur un sous-espace fermé}
Soit $H$ un espace de Hilbert. On rappelle le théorème du cours :
\begin{theo} (Projection orthogonale sur un convexe fermé)
Soit $C$ un convexe fermé non vide de $H$.
Alors,
\[
	\forall x \in H, \ \exists ! P_C(x) \in C \ / \ ||x - P_C(x)|| = \inf_{y \in C} ||x - y|| 
\]
De plus, ceci est équivalent à :
\[
	\forall y \in C, \ \Re\left<x - P_C(x), y - P_C(x)\right> \leq 0 
\]
$P_C(x)$ s'appelle alors le projeté orthogonal de $x$ sur $C$.
\end{theo}
\begin{coro} (Projection orthogonale sur un sous-espace vectoriel fermé)
Soit $F$ un sous-espace vectoriel fermé non vide de $H$.
$F$ est clairement convexe, donc $\forall x \in H, \ \exists ! P_F(x) \in F \ / \ P_F(x) = \arginf_{y \in F} ||x - y||$ et c'est équivalent à :
\[
	\forall y \in F, \ \Re\left<x - P_F(x), y\right> = 0 
\]
\end{coro}

Soit $(X, ||\cdot||_X)$ de Banach. On choisit $Y \subset X$ fermé et $x_0 \in X$.
On nous demandait si il était possible de trouver $y_1, y_2$ distincts dans $Y$ tels que $\inf_{y \in Y} ||x_0 - y||_X  = ||x_0 - y_1||_X = ||x_0 - y_2 ||_X$.
Deux exemples étaient donnés,
\begin{expl}
$X_1 = (l^2\left(\mathbb{K}\right), ||\cdot||_2)$.
\end{expl}
Dans ce cas, on sait que $X_1$ est un espace de Hilbert (cf. cours et TD).
On peut donc utiliser le corollaire précédent et en déduire l'unicité de $\arginf_{y \in Y} ||x_0 - y||_2$.
Ainsi, on ne peut pas trouver $y_1$ et $y_2$ distincts.

\begin{expl}
$X_2 = (l^1\left(\mathbb{K}\right), ||\cdot||_1)$.
\end{expl}
Nous sommes ici dans un cas clairement différent : on a démontré en cours et en TD que $(l^p\left(\mathbb{K}\right), ||\cdot||_p)$ n'est pas de Hilbert lorsque $p \neq 2$…
On ne peut donc plus utiliser le théorème précédent et, intuitivement, on se doute bien que l'on peut trouver un contre-exemple.

Prenons $\mathbb{K} = \mathbb{R}$.
Soit $B = (e_1, e_2, ...)$ la base canonique de $X_2$.
Choisissons $a$ et $b$ deux réels avec $a < b$.
Définissons le vecteur :
\[
x_0 = (a, b, 0, ...) = ae_1 + be_2
\]
Prenons maintenant 
\[
Y = \left\{y \in X_2 \ | \ y = (\alpha, \alpha, 0, ...), \ \alpha \in \mathbb{R}\right\} = \mathbb{R}.(e_1+e_2)
\]
$Y$ est-il bien fermé ?
$Y$ est un $\mathbb{R}-$evn de dimension finie, il est donc complet.
Or tout sous-espace complet est un sous-espace fermé, donc $Y$ est fermé.

Cherchons 
\[
\inf_{y \in Y} ||x_0 - y||_1
\]
Soit $y = (\alpha, \alpha, 0, ...) \in Y$.
On a :
\[
	||x_0 - y||_1 = |a - \alpha| + |b - \alpha|
\]
Posons $f(u) = ||x_0 - y||_1$.
$f$ est continue et elle est dérivable sur $\mathbb{R}\backslash\left(\{a\}\cup\{b\}\right)$.
Alors,
\[ f'(\alpha) = \left\{ 
  \begin{array}{l l l}
    -2 & \quad \text{si $\alpha < a$}\\
    0 & \quad \text{si $\alpha \in ]a,b[$}\\
    2 & \quad \text{si $\alpha > b$}
  \end{array} \right.\]
On en déduit que $f$ atteint son minimum pour tout $\alpha_{min} \in ]a,b[$ et ce minimum est $f(\alpha_{min}) = -(a - \alpha_{min}) + b - \alpha_{min} = b - a$.
On remarque qu'il est aussi atteint en $a$ et en $b$, donc sur tout l'intervalle $[a,b]$.

Ainsi, en posant par exemple $y_1 = (a, a, 0, ...)$ et $y_2 = (b, b, 0, ...)$ on a deux éléments qui sont distincts et pourtant :
\[
	\inf_{y \in Y} ||x_0 - y||_1  = ||x_0 - y_1||_1 = ||x_0 - y_2 ||_1 = b - a
\]


\section*{Remarques}
Je sais que l'analyse fonctionnelle n'est pas une matière facile, c'est pourquoi j'ai rappelé le cours et détaillé toutes les étapes de mon raisonnement au maximum.
Bien sûr, il est inutile de faire ces rappels et de détailler autant ses réponses en examen (sauf si c'est explicitement demandé).
N'hésitez pas à me contacter pour des questions/remarques (à l'adresse \href{mailto:elliot.sisteron@insa-rouen.fr}{elliot.sisteron@insa-rouen.fr}).
Bonne révisions et bonne chance pour votre futur examen,\\

Elliot.


\end{document}
